

\documentclass[11pt]{article}
%\usepackage{graphicx}
\usepackage{amssymb}
%\usepackage[]{epsfig}
%\usepackage{parskip}


\textheight 8.0in
\topmargin 0.0in
\textwidth 6.0in
\oddsidemargin 0.25in
\evensidemargin 0.25in

\newcommand{\gev}{\mathrm{GeV}}
\newcommand{\gevc}{\mathrm{GeV/c}}
\newcommand{\gevccgevcc}{\mathrm{GeV^2/c^4}}

\begin{document}


\title{Notes on the {\tt GlueX} {\tt FSRoot} Format}
\author{Ryan Mitchell}
\date{\today}
\maketitle

\abstract{This document describes the {\tt GlueX} {\tt FSRoot} format.}


\tableofcontents

\parindent 0pt
\parskip 10pt

\section{Overview}

The {\tt GlueX} {\tt FSRoot} format is a flat {\tt TTree} format.  All variables are {\tt float}.  Multiple combinations within an event are listed as separate {\tt TTree} entries, just like entries from distinct events.  The format is designed to be compatible with the {\tt FSRoot} package.


\section{Final States}
\label{sec:numbering}

The {\tt FSRoot} format can hold information from any final state composed of a combination of $\Lambda$~(decaying to $p\pi^-$), $\overline\Lambda$~(decaying to $\overline{p}\pi^+$), $e^+$, $e^-$, $\mu^+$, $\mu^-$, $p$, $\bar{p}$, $\eta$~(decaying to $\gamma\gamma$), $\gamma$, $K^+$, $K^-$,
$K^0_S$~(decaying to $\pi^+\pi^-$), $\pi^+$, $\pi^-$, and $\pi^0$~(decaying to $\gamma\gamma$). 

Final state particles are listed in trees in the following order:  
\begin{verbatim}
    Lambda ALambda e+ e- mu+ mu- p+ p- eta gamma K+ K- Ks pi+ pi- pi0
\end{verbatim}
For example, in the process $\gamma p\to\pi^+\pi^-J/\psi p; J/\psi\to\mu^+\mu^-$, the $\mu^+$ is particle~1, the $\mu^-$ is particle~2, the $p$ is particle~3, the $\pi^+$ is particle~4, and the $\pi^-$ is particle~5. Or as another example, in the process $\gamma p\to\gamma\chi_{c1}p; \chi_{c1}\to\eta\pi^0\pi^0$, the $p$ is particle~1, the $\eta$ is particle~2, the $\gamma$ is particle~3, one $\pi^0$ is particle~4, and the other $\pi^0$ is particle~5.  In cases like this where there are identical particles, no ordering is assumed.



\section{Event Information}

The ``Event Information'' variables contain information about the event as a whole:

\begin{verbatim}
  Run:          run number
  Event:        event number
  Chi2:         chi2 of the kinematic fit
  Chi2DOF:      chi2/dof of the kinematic fit
  RFTime:       RF time determined from the kinematic fit
  RFDeltaT:     difference between RF time and beam photon time
  EnUnusedSh:   total energy of unused showers
  ProdVx:       production vertex parameters
  ProdVy:         "
  ProdVz:         "
  ProdVt:         "
  PxPB:         kinematically fit beam parameters
  PyPB:           "
  PzPB:           "
  EnPB:           "
  RPxPB:        measured beam parameters
  RPyPB:          "
  RPzPB:          "
  REnPB:          "
  MCPxPB:       thrown beam parameters (MC only)
  MCPyPB:         "
  MCPzPB:         "
  MCEnPB:         "
\end{verbatim}



\section{Particle Four-Momenta}
\label{sec:nt:momentum}

The ``Particle Four-Momenta'' variables contain the four-momentum for each particle in the final state:

\begin{verbatim}
  (prefix)PxP(n):  x momentum of particle (n)
  (prefix)PyP(n):  y momentum of particle (n)
  (prefix)PzP(n):  z momentum of particle (n)
  (prefix)EnP(n):  energy of particle (n)
\end{verbatim}

Different types of four-momenta are distinguished using prefixes.  Raw four-momenta have a prefix~{\tt R}; the final four-momenta (the fully-constrained four-momenta resulting from the kinematic fit) have no prefix; MC truth information have a prefix~{\tt MC}.

Different particles are differentiated using the postfix {\tt P(n)}, where {\tt (n)} is the number of the particle in the ordered list.  Four-momenta for secondaries originating from particle~{\tt (n)}, such as the two $\gamma$'s from a $\pi^0$, are recorded using {\tt P(n)a} and {\tt P(n)b}, where the ordering follows the same conventions as above, or, in the case of identical daughter particles, no ordering is assumed.  As two examples: in the process $\gamma p \to\pi^+\pi^-J/\psi p; J/\psi\to\mu^+\mu^-$, the raw energy of the $\pi^+$ is given by {\tt REnP4}; and in the process $\gamma p \to \pi^+\pi^-J/\psi p; J/\psi\to\pi^+\pi^-\pi^0$, the y-momentum of a photon from the $\pi^0$ decay, after the kinematic fit, is given by {\tt PyP6b}.




\section{Track Information}
\label{sec:nt:track}

Track information is written out for every reconstructed track that is part of a final state.  The postfix {\tt P(n)} follows the same convention as for the four-momenta (section~\ref{sec:nt:momentum}).
\begin{verbatim}
  TkChi2P(n):   chi2 of the track fit
  TkNDFP(n):    number of degrees of freedom for the track fit
\end{verbatim}


\section{Shower Information}
\label{sec:nt:track}

Shower information is written out for every reconstructed shower that is part of a final state.  The postfix {\tt P(n)} follows the same convention as for the four-momenta (section~\ref{sec:nt:momentum}).
\begin{verbatim}
  ShQualityP(n):   a shower quality score
\end{verbatim}


\section{MC Truth Information}
\label{sec:nt:track}

The following variables can be used to help keep track of truth information.

\begin{verbatim}
  MCDecayCode1:      the true code1, described in the
                       "final state numbering" section
  MCDecayCode2:      the true code2, described in the
                       "final state numbering" section
  MCExtras:          1000 * number of neutrinos +
                      100 * number of K_L +
                       10 * number of neutrons +
                        1 * number of antineutrons
  MCSignal:          1 if the reconstructed final state
                       matches the generated final state;
                     0 otherwise
\end{verbatim}


\section{Final State Numbering}
\label{sec:numbering}

Final states are designated using two integers, ``code1'' and ``code2''.  The digits of each integer are used to specify the number of different particle types in the final state:
\begin{verbatim}
    code1 = abcdefg
            a = number of gamma
            b = number of K+ 
            c = number of K- 
            d = number of Ks  ( --> pi+ pi- )
            e = number of pi+ 
            f = number of pi- 
            g = number of pi0 ( --> gamma gamma )

    code2 = hijklmnop
            h = number of Lambda (--> p+ pi-)
            i = number of ALambda (--> p- pi+)
            j = number of e+
            k = number of e- 
            l = number of mu+ 
            m = number of mu-
            n = number of p+ 
            o = number of p- 
            p = number of eta  ( --> gamma gamma )
\end{verbatim}
These integers are sometimes combined into a single string of the form:
\begin{verbatim}
    "code2_code1"
\end{verbatim}
Here are a few examples:
\begin{verbatim}
    "0_111":      pi+ pi- pi0
    "0_1000002":  gamma pi0 pi0
    "1_220000":   eta K+ K+ K- K-
    "11000_110":  mu+ mu- pi+ pi-
\end{verbatim}
%In the root trees, four-momenta and other information are listed for particles in the following order:
%\begin{verbatim}
%    e+ e- mu+ mu- p+ p- eta gamma K+ K- Ks pi+ pi- pi0
%\end{verbatim}



\end{document}
